\chapter{Testing PHP and JavaScript Code with automated Tools}

\section{Testing PHP Code}

First of all, the php code is very structured and readable. Although that makes it easy to understand what the code is doing, we couldn't find big vulnerabilities in the php code.
We used  the two automated tools RATS and RIP for testing the PHP code. While Rats doesn't find any vulnerabilities at all, RIPS is claiming some including errors, shown in Listing .
\begin{lstlisting}[caption= Output of automated testing tool RIPS]
Include error: tried to include: /Phase4/secure-coding/backend/rest/include/libphp-phpmailer/class.smtp.php
9: require require ("libphp-phpmailer/class.smtp.php"); 

Include error: tried to include: /Phase4/secure-coding/backend/rest/include/libphp-phpmailer/class.pop3.php
10: require require ("libphp-phpmailer/class.pop3.php"); 

Include error: tried to include: /Phase4/secure-coding/backend/rest/include/libphp-phpmailer/class.phpmailer.php
11: require require ("libphp-phpmailer/class.phpmailer.php"); 
\end{lstlisting}
This errors are not leading to vulnerabilities. The reason for the lack of found vulnerabilities besides the quality of the code, is the fact, that it's object oriented. Both tools are clearly stating that they don't support object oriented php code yet. So these results are no big surprise.

\section{Testing JavaScript Code}
Since the application uses  the AngularJS framework to enhance html, we decided to test the javascript code in terms of security. Therefore we used two Testing Tools: JSPrime and ScanJS




 